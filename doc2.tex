\documentclass[12pt]{book}
\usepackage{amsmath}

\begin{document}

\title{Mathematical typesetting}
\author{silpa}
\maketitle

\chapter{equations}
\section{Inline Mode}
Linear equation: $ x+y=0 $ and $ x^{31}$

\section{Displayed Mode}
\underline{unnumbered}\\
In physics,the mass energy equivalence is started by the equation $$ E=mc^2 $$discovered in 1985\\ \\

\underline{numbered}\\ \\
In natural units,
\begin{displaymath}
c=1
\end{displaymath}

The formula identified
\begin{equation}
c=a+b
\end{equation}
\section{Aligning Equation}
\begin{align*}
x+3y+4z&=2 \\
3y-4z&=5\\
3&=4
\end{align*}
\section{fraction}
$$ \frac{x+3}{4} $$
\section{subscript and suprscript}
$$ a_{10} $$ 
$$ a_1 $$
$$ x=\frac{-b \pm \sqrt{b^2-4ac}}{2a} $$

\section{special characters}
\{ \& \}

\section{matrices}
1,plain\\
$$
\begin{matrix}
1 & 2 & 3 \\
A & B & C
\end{matrix}
$$ \\
$$
\begin{pmatrix}
1 & 2 & 3 \\
A & B & C
\end{pmatrix}
$$
$$
\begin{bmatrix}
1 & 2 & 3 \\
A & B & C
\end{bmatrix}
$$\\

$$
\begin{vmatrix}
1 & 2 & 3 \\
A & B & C

\end{vmatrix}
$$
\section{Dispaly style}
$f(x)=\frac{1}{1+x} $ can be set with a different style:$ f(x)=\displaystyle\frac{1}{1+x} $
\end{document}

